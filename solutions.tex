\documentclass{scrartcl}
\usepackage[ngerman]{babel}
\usepackage[utf8]{inputenc}
\usepackage[T1]{fontenc}
\usepackage{lmodern}
\usepackage[dvipsnames,svgnames]{xcolor}
\usepackage{graphicx}
\usepackage{enumerate}
\usepackage{multicol}
\usepackage{float}
\usepackage{hyperref}
\definecolor{darkblue}{rgb}{0,0,.5}
\hypersetup{pdftex=true, colorlinks=true, breaklinks=true, linkcolor=darkblue, menucolor=darkblue, pagecolor=darkblue, urlcolor=darkblue}

\usepackage{amsmath, amssymb, stmaryrd}
\usepackage{amsthm, mathtools}

\theoremstyle{definition}
\newtheorem{exercise}{Task}

\usepackage{tikz}
\title{Assignment Sheet 2}
\author{Timm Behner \and Christopher Kannen \and Eva-Lotta Teutrine}
\date{\today}

\begin{document}
\maketitle
\setcounter{exercise}{2}
\begin{exercise}
    Let $\overline m$ be an arbitrary message in blocks of length 8, hence
    $\overline m = \left( \overline m_1, \dots , \overline m_t \right)$. Then define
    \begin{align}
        y_i = \begin{cases}
            0 & \text{ if } K(m)_i = K(\overline m)_i \\
            1 & \text{ if } K(m)_i \neq K(\overline m)_i \\
        \end{cases}
    \end{align}
    where $K(m)_i$ denotes the $i$th bit of the 64-bit value $K(m)$. Let $m' =
    \left( \overline m_1, \dots , \overline m_t, y \right)$.
    Then $K(m) = K(\overline m) \oplus y = K(m')$ 
    From this follows $H_k(m') = E_k(K(m)) = E_k(K(m')) = H_k(m)$ which can be
    known without the knowledge of the key $k$.
\end{exercise}
\end{document}
